\section{Einleitung}
\subsection{Anreger \& Betreuer}
Auftraggeber und Betreuer dieser Arbeit war AO. Univ. Prof. Dr. Erich Neuwirth.
Ich m"ochte ihm hier an dieser Stelle bei ihm f"ur die ausgezeichnete Betreuung
und Zusammenarbeit bedanken.

\subsection{Ziel der Arbeit}
Die vorliegende Arbeit versteht sich als Arbeitsunterlage f"ur Lehrpersonen.
Nicht jedoch aber als Handout f"ur die Lernenden. Lehrende sollen durch dieses
Dokument einen Einblick in die verwendeten Tools bekommen und aufbereitete
Beispiele vorfinden, welche sie f"ur ihren Unterricht verwenden k"onnen.

Der Einsatz von Musik im Informatikunterricht mag auf den ersten Blick etwas
seltsam, wenn nicht sogar verst"orend wirken. Er bringt jedoch einen gro"sen
Vorteil mit sich: Fehler im Ansatz oder in der Ausarbeitung sind sofort
h"orbar. Ein falsch transponierter Dreiklang kann auch von einem musikalisch
nicht begabten Lernenden erkannt werden. Zu dem kann die Musik als m"achtiges
didaktisches Werkzeug gesehen werden, wenn es um die Vermittlung von parallelen
Verarbeitungsmodellen geht. Die Metapher eines Orchesters kann sehr gut heran
gezogen werden, da in ein Orchester aus mehreren MusikerInnen besteht. Diese
agieren von einander im wesentlichen unabh"angig, m"ussen jedoch bis zu einem
gewissen Grad von etwas bzw. jemand gesteuert werden: Dem Dirigenten, auf
englisch dem Condcutor. Auch die Metapher des Dirigenten/Conductor wird in 
meiner Arbeit stark zum Einsatz kommen. 

\emph{Wichtig f"ur einen Einsatz im Unterricht, ist der Hinweis, dass die
Lernenden Kopfh"orer mitbringen sollten.} Spielen mehrere ohne Kopfh"orer
ist ein Unterricht kaum mehr m"oglich und es herrscht Chaos. 

Beispiele daf"ur, was mit dem Toolkit m"oglich ist, finden sich im letzten 
Abschnitt. Es ist beispielsweise m"oglich, 2 Rettungsautos aus verschiedenen 
Richtungen fahren zu lassen und die Signalt"one der Autos akustisch zu lokalisieren.


\subsection{Verwendete Tools}
Im wesentlichen wird die NetLogo-Extension 'midi' verwendet. Die Midi-Extension
f"ur NetLogo enth"alt einige der F"ahigkeiten die das MidiCSD-Toolkit\cite{MidiCSD} f"ur 
OpenOffice bzw. Excel 
ebenfalls beherrscht. Der n"achste Abschnitt beschreibt die Notwendigkeit
der NetLogo-Erweiterung bzw. zeigt die Anleitung in 'Installation
der Erweiterung f"ur NetLogo' wie simpel sie zu installieren ist und
somit einfach im Unterricht eingesetzt werden kann. 
Es sei an dieser Stelle explizit auf MidiCSD hingewiesen. Gerade f"ur 
Experimente mit Musik eignet es sich hervorragend um im Informatik-Unterricht
angewendet zu werden. 

\subsection{Ausgangslage f"ur NetLogo}
\lstset{language=Logo}
Bisher st"orte die Tatsache, dass es nicht m"oglich war in NetLogo Midi-Kan"ale
direkt anzusprechen. In der Version 4.1.1 (aktuell zum Erstellungszeitpunkt des 
Projektes Juli, 2010) konnten lediglich die Befehle 
\lstinline|sound:play-drum drum velocity| und 
\lstinline|sound:play-note instrument keynumber velocity duration| verwendet
werden um Modelle mit T"onen zu versehen. %Das bedeutet aber gro"se 
%Einschr"ankungen. 
Als Konsequenz daraus konnte man den einzelnen Akteuren keine musikalischen Aktionen 
zuordnen. So ist es zum Beispiel nicht m"oglich f"ur das Beispiel 
''Rettungsauto'' (siehe \ref{Bsp:Rettungsauto}) den Dopplereffekt zu erzeugen,
geschweige denn "uberhaupt den Ton des Autos lauter werden zu lassen im Falle
des sich ann"ahernden Autos und wieder leiser werden zu lassen, im Fall des sich
entfernenden Autos. 

NetLogo hat eine (fast) unbeschr"ankte Anzahl an Akteuren, den Turtles. Bisher
war es aber nicht m"oglich Musik, oder einzelne T"one mit diesen zu verbinden.
Sieht man sich die Beispielsammlung der NetLogo-Distribution an, findet man im
Bereich Musik nur wenige Beispiele. Auch kann in diesen die volle Macht von
NetLogo nicht ausgenutzt werden. So wird z.B. im Beatbox Beispiel die Musik
von einer zentralen Prozedur aus angewandt. Alle 'Trommler' spielen jedoch
auf dem selben Midi-Kanal. Das w"are also so als w"urde doch nur ein Trommler spielen
und mehrere Percussioninstrumente zur gleichen Zeit bedienen.

Im vorigen Absatz wird erw"ahnt, dass nur auf einen Midi-Kanal zugegriffen wird.
Das ist wie schon eingangs erw"ahnt ein gro"ses Manko. Es ist also nicht
m"oglich Effekte nur f"ur einzelne Instrumente zu steuern. Beispiel: Der 
Trompeter soll leise spielen weil gerade die Fl"ote ein Solo hat (Funktioniert
im echten Leben leider auch nicht immer). 

\subsection{Installation der Erweiterung f"ur NetLogo}
Es gibt zwei Varianten um die Erweiterung zu installieren
\begin{itemize}
\item die Datei ''midi.jar'' in einen Unterordner ''midi'' neben das Model legen,
welches die Erweiterung verwenden m"ochte.
\item die Datei ''midi.jar'' in einen Ordner ''midi'' in das Verzeichnis ''extensions''
der NetLogo-Installation legen. 
\end{itemize}
Ebenso erfolgt die Installation f"ur als Applet im Internet hinterlegte Modelle.
Auch hier muss die ''midi.jar'' Datei ein einem ''midi'' Unterordner liegen.

\subsection{Gliederung der Arbeit}
Ich werde zuerst kurz beschreiben wie ich NetLogo um zus"atzliche Befehle erweitert
habe. Dann werde ich die Konzepte hinter der Erweiterung erl"autern und
Beispiele bringen wie die Befehle eingesetzt werden k"onnen. Diese Beispiele
sind schon so vorbereitet um in einem etwaigen Unterricht mit Sch"ulern eingesetzt
werden zu k"onnen. %Am Ende der Arbeit befinden sich die Listings der Erweiterung. 

Es werden einige Details hinter der Arbeit sehr ausf"uhrlich erl"autert. Dies 
wird gemacht um den Lesern bzw. Anwendern dieser Arbeit einen tieferen Einblick
in die Interna der Erweiterung zu geben. Ein tieferes Verst"andnis der Arbeit kann
einen verbesserten Einsatz im Unterricht erm"oglichen, da so jeder Lehrende 
individuell St"arken und Schw"achen f"ur den pers"onlichen Einsatz im Unterricht
ausmachen kann. Weiters tauchen in der Praxis der Informatik immer wieder Probleme
auf die nur durch ein detailliertes Wissen der verwendeten Mittel gel"ost werden
k"onnen. Auch r"ustet ein gutes Hintergrundwissen "uber die Erweiterung die
Lehrenden f"ur etwaige Fragen von interessierten Lernenden, welche zB die im
Unterricht erl"auterten Beispiele noch erweitern wollen oder bei der Erweiterung
auf Probleme gesto"sen sind. 

\subsection{Test der Erweiterung}
Da ein Teil meiner Arbeit war NetLogo\cite{NetLogo} Midi tauglich zu machen,
musste dies auch getestet werden. 
Die Arbeit wurde unter anderem mittels der weiter hinten in der Arbeit 
beschriebenen Beispiele
getestet. Es ist also gew"ahrleistet, dass diese Beispiele in der erl"auterten
Form funktionieren und keine Seiteneffekte hervorrufen. 
Getestet wurde sowohl auf Windows PCs so wie auf Apple Computern. 
Verwendet wurde die zum Zeitpunkt der Arbeit aktuelle NetLogo Version 4.1.1. 

Es wurden bei den Tests keine Unterschiede zwischen Windows und Mac OS entdeckt.

% \subsection{Einsatz in der Unterrichtspraxis}
% Die vorliegende Erweiterung kann direkt f"ur Unterrichtszwecke verwendet werden.
% Bei der Ausarbeitung der Beispiele weiter hinter hinten in der Arbeit wurde
% darauf geachtet sie m"oglichst gut aufzubereiten um sie direkt in Schulungen
% einsetzen zu k"onnen. Sie liegen liegen als fertig implementierte Modelle vor. 
% Die Modelle sind nicht direkt Teil meiner Arbeit. Sie wurden von Prof. Dr. Erich
% Neuwirth erdacht und ich habe sie in seiner Lehrveranstaltung ''Kernthemen der
% Fachdidaktik Informatik'' im Wintersemester 2009 kennen gelernt. 

\subsection{W"unschenswertes f"ur die NetLogo-Erweiterung}
Die Erweiterung wurde mit der Version 4.1.1 von NetLogo entwickelt. In geraumer
Zeit sollte die Version 4.1.2 fertig gestellt werden. Mit dieser Version kann
ein Problem meiner Erweiterung eventuell gel"ost werden und das hinzuf"ugen von
Befehlen zu Notenbl"attern direkt als Befehl m"oglich sein. Es seien hier einige
Aspekte aufgef"uhrt die von mir nicht in die Erweiterung aufgenommen wurden,
manch Lehrendem aber als fehlende Aspekte auffallen. Diese Liste erhebt aber keinen
Anspruch auf Vollst"andigkeit. Feedback von Lehrenden ist erw"unscht!

Die aktuelle Implementierung k"onnte noch in folgenden Punkten erweitert werden:
\begin{itemize}
\item \emph{Midi-Rendern} Es k"onnen zur zeit nur direkt Befehle ausgegeben 
werden. Eine m"ogliche Erweiterung ist, dass anstatt die Befehle auszugeben,
die akustische Ausgabe des Modells in ein Midi-File zu schreiben. 

\item \emph{Midi-Compiler} Wie sp"ater hinter in der Arbeit angemerkt wird, hat
die Implementierung Befehlsabarbeitung noch Potential zur Optimierung. Es k"onnte
angedacht werden, einzelne Bl"atter als Midi-Bl"atter zu deklarieren. Zu diesen
k"onnen dann nur Midi-Befehle hinzugef"ugt werden, welche dann direkt als javainterne
Midi-Commands kompiliert werden.

\item \emph{Protokollierung} Das Projekt k"onnte zus"atzlich um die Funktionalit"at
von Protokollausgaben der Befehle erweitert werden k"onnen. Anstatt von Midi-Noten
werden k"onnten dann normale Notennamen ausgegeben werden k"onnen. 

\end{itemize}

