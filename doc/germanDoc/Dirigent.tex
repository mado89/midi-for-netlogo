\section{Der Dirigent in Midi-NetLogo}
Ein weiterer Punkt den ich in die Aufgabenstellung inkludiert habe, war die
M"oglichkeit Sequenzen zu definieren, welche dann unabh"angig abspielbar sein 
sollen. Dieser Punkt wurde dann erweitert, dass es m"oglich sein soll, alle
in NetLogo verf"ugbaren Befehle in so eine Sequenz zu packen. 

Das kann nun daf"ur verwendet werden um im Informatik-Unterricht Threads zu
vermitteln. Ein Thread ist damit nichts anderes als ein Sheet. 

\subsection{Konzept}
Wie der Name schon andeutet, ist dieser Aspekt mit der Metapher eines Orchesters
gel"ost worden. Ein Dirigent hat die Kontrolle "uber mehrere Musiker die 
Aktionen ausf"uhren sollen. Er bestimmt wann das Orchester zu spielen beginnt
und wann es wieder aufh"ort. Weiters kann er nat"urlich auch wieder von Vorne
beginnen lassen.

Jedes Programm braucht auch Mechanismen zur Steuerung und Synchronisation 
seiner Threads. Der Dirigent "ubernimmt diese Aufgabe und erm"oglicht so eine
vereinfachte Form der Vermittlung von Thread-Synchronisation. 

\subsection{Technische Realisierung}
Wie schon im Konzept erw"ahnt wird NetLogo durch die Midi-Extension eine Zentrale Instanz 'Conductor'
beigebracht. Diese verf"ugt "uber die folgenden F"ahigkeiten: \begin{itemize}
\item Alle Notenbl"atter l"oschen
\item Zu einem Notenblatt etwas hinzuf"ugen
\item Zu einem Notenblatt einen mit Timecode versehenen Befehl hinzuf"ugen
\item Alles auf Anfang setzen
\item Dirigieren
\item Endlos spielen lassen, normal spielen lassen
\end{itemize}

\subsubsection{Sheets - Kan"ale}\label{sec:sheet-channel}
Der Dirigent h"alt eine fixe Anzahl von 16 Sheets. Die Zahl ist so festgelegt,
da Midi 16 Kan"ale besitzt. Die Midi-Kan"ale besitzen die Nummern 1 - 16, die
Sheets jedoch die Indizes/Nummern 0 - 15. "Uber eine Variable f"ur Agents kann
eine Zuordnung von Agenten zu einem Kanal gemacht werden. 
\begin{lstlisting}[language=Logo]
turtles-own [channel]
...
to init
  den Turtles channels zu ordnen
end

...

to some.procedure
	ask turtles[
		midi:instrument channel 60
	]
end
\end{lstlisting}
Eine sehr einfach Variante den Turtles Kan"ale zuzuornen ist:
\begin{lstlisting}[language=Logo]
ask turtles[
	set channel who + 1
]
\end{lstlisting}

\subsubsection{Events}
Die zentrale Aufgabe der Sheets ist es Befehle in chronologischer Reihenfolge
auszuf"uhren. Es gibt zwei Befehle um einem Notenblatt Befehle hinzu zuf"ugen:
\begin{itemize}
\item conductor.add.to.sheet
\item conductor.add.to.sheet.list
\end{itemize}
Beide Befehle machen im wesentlichen das Gleiche, jedoch werden dem Zweiten
eine Liste von mit einem Timecode versehenen Befehlen "ubergeben, ersterer f"ugt
einen einzelnen Befehl versehen mit einem Timecode zu einem Notenblatt hinzu. 
Eine genauere Beschreibung ist im Kapitel "uber die Befehle zu finden. 



