\section{Befehle der NetLogo-Extension}
Alle Befehle beginnen mit einem vorangestellten \lstinline|midi:|. 
Ich m"ochte hier kurz die Befehle beschreiben um der Lehrperson einen tieferen 
Einblick zu gew"ahren und damit diese den Lernenden bei etwaigen Problemen kompetent
helfen k"onnen. 

Die Dokumentation der Befehle erfolgt immer in der Form \lstinline|Befehl Parameter|
was dann in einem NetLogo-Programm als \lstinline|midi:Befehl Parameter| geschrieben
wird. Die Befehlsnamen sind nicht(!) casesensitive. 
\subsection{Midi-Befehle}
\subsubsection{Standard Befehle}
Auf eine Dokumentation der Standard-Midi-Befehle m"ochte ich hier verzichten. Es sei
auf das Studium einschl"agiger Fachliteratur verwiesen. 
Folgende Befehle sind in der Erweiterung verf"ugbar: 
noteon,
noteoff,
instrument (F"ur eine Liste an Instrumenten siehe \cite{midi-inst}),
pitch.bend,
controller,
key.pressure,
channel.pressure,
volume,
expression,
modulation,
pan,
sustain,
reverb,
chorus,
portamento.time,
portamento,
portamento.from,
rpn,
nrpn,
reset.controllers,
all.notes.off,
pitch.sens channel,
mastertune.coarse,
mastertune.fine,
panic
% \subsubsection{Standard Befehle}
% \begin{itemize}
% \item \lstinline|noteon channel note volume|
% \item \lstinline|noteoff channel note|
% \item \lstinline|instrument channel instrument| (F"ur eine Liste an Instrumenten
% siehe \cite{midi-inst}
% \item \lstinline|pitch.bend channel bend|
% \item \lstinline|controller channel controller parameter|
% \item \lstinline|key.pressure channel note pressure|
% \item \lstinline|channel.pressure channel pressure|
% \item \lstinline|volume channel volume|
% \item \lstinline|expression channel expression|
% \item \lstinline|modulation channel modulation|
% \item \lstinline|pan channel pan|
% \item \lstinline|sustain channel value|
% \item \lstinline|reverb channel note volume duration|
% \item \lstinline|chorus channel duration|
% \item \lstinline|portamento.time channel time|
% \item \lstinline|portamento channel onoff|
% \item \lstinline|portamento.from channel note|
% \item \lstinline|rpn channel lorpn hirpn lodata hidata|
% \item \lstinline|nrpn channel lorpn hirpn lodata hidata|
% \item \lstinline|reset.controllers channel|
% \item \lstinline|all.notes.off channel|
% \item \lstinline|pitch.sens channel semitones|
% \item \lstinline|mastertune.coarse channel hidata|
% \item \lstinline|mastertune.fine channel cents|
% \item \lstinline|panic|
% \end{itemize}
\subsubsection{Der Note Befehle}
\lstinline|note channel note volume duration| \label{note-warning}


\begin{emph}Achtung: Dieser
Befehl ist ein Blocking-Call. Er spielt eine Noten in der Dauer von 
\lstinline|duration| (in ms angegeben ab). Erst dann wird das Programm fortgesetzt.
Er setzt zuerst die Note und legt den Thread dann schlafen, nach Abwarten des
Notenwertes l"oscht er die Note vom entsprechenden Midi-Kanal. Dieses Verhalten
kann zu Problemen f"uhren wenn sie in Programmen in Kombination mit den 
Conductor-Utilities verwendet werden. \end{emph}
\subsection{Befehle f"ur Turtles/Agenten}
\lstinline|updatepostion channel| 

Dieser Befehl setzt eine eine Position im akustischen Raum f"ur die aktuelle
Turtle, kann also nur in einem Turtle-Context angewandt werden. Die Position
wird f"ur die Turtle die den Befehl aufruft errechnet und dann als virtuelle
Position auf den "ubergebenen Channel gelegt (siehe auch \ref{sec:sheet-channel}. 
Ver"andert werden die Parameter Pan und Expression. Ausgangspunkt f"ur die
Berechnung ist der Koordinaten-Ursprung des NetLogo-Models. Als maximaler Wert
f"ur die Skalierung wird auch jeweils der maximale Wert des Koordinatensystems
verwendet wird. Das hei"st wenn das Model zB. auf der positiven X-Achse mehr Werte
zul"asst als auf der negativen. Ist die Abstufung auf der positiven Seite feiner.

\subsection{Befehle f"ur den Condcutor}
\subsubsection{clear.sheets}
Syntax: \lstinline|conductor.clear.sheets|

L"oscht alle dem Conductor bekannten Notenbl"atter.
\subsubsection{add.to.sheet}
Syntax: \lstinline|conductor.add.to.sheet sheet time.distance command|

Dieser Befehl f"ugt dem angegebenen Notenblatt am Ende ein Kommando hinzu. Das
Kommando wird nach Ablauf der durch \lstinline|time.distance| angegebenen Zeit
(in ms) ausgef"uhrt. Zus"atzlich ist das der Zeitpunkt von welchem aus der 
Zeitpunkt f"ur ein m"ogliches nachfolgendes Kommando berechnet wird. Beispiel:
\begin{lstlisting}[language=Logo]
midi:conductor.clear.sheets
midi:conductor.add.to.sheet 1 10 "midi:noteon 2 60 1"
midi:conductor.add.to.sheet 1 2000 "midi:noteoff 2 60"
midi:conductor.start
\end{lstlisting}
Nach Ablauf von zehn Milisekunden wird vom Notenblatt Eins auf dem Midi-Kanal
Zwei die Note mit dem Wert 60 gespielt. Nach zwei Sekunden, also nach insgesamt
2010 Milisekunden verstummt die Note wieder. 

\subsubsection{add.to.sheet.list}
Syntax: \lstinline|conductor.add.to.sheet.list sheet command.list|

Dieses Kommando f"ugt dem angegebenen Notenblatt am Ende die Kommandos aus der 
Liste ein. Die Kommando-Liste hat eine einfache Syntax: \lstinline|[time.code command]|
Die Time-Codes funktionieren analog zu denen des \lstinline|add.to.sheet| Befehls.
Das Kommando muss aber als String angegeben werden! Das kommt daher, dass NetLogo
zum Entwicklungszeitpunkt (Juli-Oktober 2010) nicht ausreichend mit dem Syntax-Typ
''Command'' umgehen konnte. Die Befehle sollten also getestet werden bevor sie
auf die Notenbl"atter geschrieben werden. Der Vorteil dieses Befehls ist, dass
er, mit geringen Modifikationen, die Ausgabe des ''to Logo'' des MidiCSD-Toolkits \cite{MidiCSD}
f"ur Office Dr. Erich Neuwirth verwendet kann. Das untenstehende Beispiel demonstriert,
wie der Befehl eingesetzt werden kann. 
\begin{lstlisting}[language=Logo]
to trommler
  clear-all
  midi:conductor.clear.sheets
 
  midi:conductor.add.to.sheet.list 1 [
		[0 "midi:note 10 45 1 200"]
		[200 "midi:note 10 45 0.7 200"]
		[200 "midi:note 10 45 0.7 200"]
		[200 "midi:note 10 45 0.7 200"]
		[200 "midi:note 10 45 1 200"]
  ]
  midi:conductor.add.to.sheet.list 2 [
		[200 "midi:pan 10 -0.75"]
		[200 "midi:pan 10 -0.5"]
		[200 "midi:pan 10 -0.25"]
		[200 "midi:pan 10 0"]
  ]
  
  midi:conductor.add.to.sheet.list 3 [
		[5 "midi:expression 10 0.75"]
		[200 "midi:expression 10 0.69"]
		[200 "midi:expression 10 0.63"]
		[200 "midi:expression 10 0.56"]
		[200 "midi:expression 10 0.5"]
  ]
  
  midi:conductor.setplaymode.endless

  midi:conductor.start
end
\end{lstlisting}

\subsubsection{restart}
Syntax: \lstinline|conductor.restart|

Die Position auf den Notenbl"attern wird wieder auf Null gesetzt. Alle Notenbl"atter
werden also wieder von Vorne abgearbeitet. 
\subsubsection{conduct}
Syntax: \lstinline|conductor.conduct|

Der Dirigent wird angewiesen die aktuell anfallenden Befehle auf den Notenbl"attern
abzuarbeiten. 

\subsubsection{setplaymode.endless}
Syntax: \lstinline|conductor.setplaymode.endless|

Erreicht ein ''Musiker'' auf seinem Notenblatt das Ende, soll er wieder von 
Vorne beginnen. Anders gesagt: Die Notenbl"atter sind nun nicht mehr linear
sondern Kreise. Auf unterschiedliche L"angen/Dauer der Notenbl"atter wird aber
nicht geachtet. 

\subsubsection{setplaymode.normal}
Syntax: \lstinline|conductor.setplaymode.normal|

Anders als beim Endlos-Playmode wird der Dirigent angewiesen, wenn ein Notenblatt
zu Ende ist, nicht wieder von Vorne zu beginnen sondern aufzuh"oren. 
\subsubsection{start}
Syntax: \lstinline|conductor.start|

Dieser Befehl wei"st den Dirigenten an im Hintergrund zu dirigieren. Das Programm
l"auft weiter w"ahrend im Hintergrund die Notenbl"atter abgearbeitet werden. 
(Siehe auch \lstinline|conductor.stop|)
\subsubsection{stop}
Syntax: \lstinline|conductor.stop|

Dieser Befehl weist den Dirigenten an seine Arbeit zu beenden. Die Abarbeitung
der Notenbl"atter im Hintergrund wird beendet. 
